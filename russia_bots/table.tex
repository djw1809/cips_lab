%TODO:
% - Make an "Exercise/solution" environment - restructure everything in terms of that
% - Maybe add all my expanded proofs of the lemmas from my notes.



\documentclass[a4paper, titlepage]{article}



%packages
\usepackage[english]{babel}
%\usepackage[utf8x]{inputenc}
\usepackage[T1]{fontenc}
\usepackage{amsmath}
\usepackage{graphicx}
\usepackage{pgfplots}
\usepackage{dsfont}
\usepackage[colorinlistoftodos]{todonotes}
\usepackage{braket,mleftright}
\usepackage{subcaption}
%\usepackage{ upgreek }
\usepackage{tikz}
\usepackage[labelfont=bf]{caption}
\usetikzlibrary{arrows}
\usetikzlibrary{decorations.markings}
\usepackage{amssymb}
%\usepackage{amsthm}
\usepackage{etoolbox}
\AtBeginEnvironment{quote}{\singlespacing\small}

\usepackage{enumitem}  %%good lists


%%macros
%%To make a better disjoint union symbol
\makeatletter
\def\moverlay{\mathpalette\mov@rlay}
\def\mov@rlay#1#2{\leavevmode\vtop{%
   \baselineskip\z@skip \lineskiplimit-\maxdimen
   \ialign{\hfil$\m@th#1##$\hfil\cr#2\crcr}}}
\newcommand{\charfusion}[3][\mathord]{
    #1{\ifx#1\mathop\vphantom{#2}\fi
        \mathpalette\mov@rlay{#2\cr#3}
      }
    \ifx#1\mathop\expandafter\displaylimits\fi}
\makeatother
%%spaces
\newcommand{\Rn}{\mathds{R}^{n}}
\newcommand{\Rd}{\mathds{R}^{d}}
\newcommand{\R}{\mathds{R}}
\newcommand{\Q}{\mathds{Q}}
\newcommand{\C}{\mathds{C}}
\newcommand{\W}[1]{\mathbf{W}^{1,#1}}
\newcommand{\Lp}{L^{p}(U)}
\newcommand{\Lq}[1]{L^{#1}(U)}

%%integrals
\newcommand{\pcint}[2]{\text{p.c.}\int_{#1}^{#2}} %piecewise constant integral
\newcommand{\ldbint}[2]{\underline{\int_{#1}^{#2}}} %lower darboux integral
\newcommand{\udbint}[2]{\overline{\int_{#1}^{#2}}}

%measure theory
\newcommand{\upmeas}[1]{m^{*}(#1)}
\newcommand{\meas}[1]{m(#1)}
\newcommand{\indi}[1]{\mathds{1}_{#1}}

%%misc
\newcommand{\norm}[1]{\left\lVert#1\right\rVert}
\newcommand{\vect}[1]{\mathbf{#1}}
\newcommand{\cupdot}{\charfusion[\mathbin]{\cup}{\cdot}}
\newcommand{\bigcupdot}{\charfusion[\mathop]{\bigcup}{\cdot}}


%%graph drawing environment
\oddsidemargin 0in \evensidemargin 0in \textwidth 6.5in \topmargin
0in \headheight 0in \headsep 0in \textheight 9in

\newcommand{\graphnode}[1]{\draw[fill] #1 circle [radius=.1]}

\newcommand{\edge}[2]{\draw[postaction={decorate}] #1 -- #2;}

\newenvironment{graphz}{\begin{scope}[very thick,decoration={
    markings,
    mark=at position 0.6 with {\arrow[black]{stealth}}}
    ]}{\end{scope}}

\newcommand{\leftedge}[2]{\draw[postaction ={decorate}, bend left] #1 to #2;}

\newcommand{\rightedge}[2]{\draw[postaction ={decorate}, bend right] #1 to #2;}




%%theorems
\usepackage[thmmarks,  thref, amsmath]{ntheorem}

%%Standard environments
\theoremstyle{plain}
\newtheorem{defn}{Definition}
\newtheorem{lem}{Lemma}
\newtheorem{thm}{Theorem}

%%Case environments/problem part environments - only to be used inside of a proof/solution environment
\theoremheaderfont{\itshape}
\theorembodyfont{\upshape}
\newtheorem{case}{Case}

\theoremstyle{break}
\theoremheaderfont{\scshape\bfseries}
\theorembodyfont{\upshape}
\theoremsymbol{}
\newtheorem{ppart}{Part}

%%proof environments
\theoremstyle{nonumberbreak}
\theoremheaderfont{\scshape\bfseries}
\theorembodyfont{\upshape}
\theoremsymbol{\rule{0.7em}{0.7em}}%
\theorempostwork{\setcounter{case}{0}\setcounter{ppart}{0}}
\newtheorem{soln}{Solution}
\newtheorem{Proof}{Proof}


%%notes/remakrs/examples environments
\theoremstyle{nonumberplain}
\theoremheaderfont{\itshape\bfseries}
\theorembodyfont{\upshape}
\theoremsymbol{}
\newtheorem{note}{Note}
\newtheorem{claim}{Claim}
\newtheorem{cust}{}
